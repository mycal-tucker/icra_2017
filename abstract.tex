This paper addresses the grounding problem whose objective is to understand the meaning of natural language commands within a robot's symbolic world. Existing techniques to solve the grounding problem typically assume that there is a fixed set of phrases or objects that the robot will encounter during deployment. However, the real world is full of confusing jargon and unique objects that are nearly impossible to anticipate and therefore train for. This paper introduces a model called the Distributed Correspondence Graph - Unknown Phrase, Unknown Percept - Away that enables to explicitly represent unknown phrases and objects as unknown, as well as reasoning about objects that are not currently perceived. The effectiveness of the model is evaluated in terms of grounding and learning new phrases via simulations and real experiments.

%Research in automatic natural language grounding, the problem of robots associating phrases with realworld objects and actions, offers a tantalizing reality in which untrained humans can operate sophisticated robots. 

%Current techniques for training robots to understand natural language, however, assume that there is a fixed set of phrases or objects that the robot will encounter during deployment. 

%Instead, the real world is full of confusing jargon and unique objects that are nearly impossible to anticipate and therefore train for. 
%This paper presents a model called the Distributed Correspondence Graph - Unknown Phrase, Unknown Percept - Away (DCGUPUP- Away) that augments a previously successful model by explicitly representing unknown phrases and objects as unknown, as well as reasoning about objects that are not currently perceived. 
%Furthermore, experimental results in simulation,
%as well as a trial run on a physical turtlebot, validate the
%effectiveness of DCG-UPUP-Away in grounding and learning
%new phrases.