This work addresses the symbol grounding problem, that is, understanding the ``meaning" of natural language within a robot's workspace. The state of the art models used in the grounding problem typically assume a fixed set of phrases or objects that are defined a priori to mission. However, the real world is full of unexpected objects that are nearly impossible to anticipate and therefore train for. This paper proposes a model called the ``Distributed Correspondence Graph - Unknown Phrase, Unknown Percept - Away" that explicitly represents unknown phrases and objects as unknown symbols and enables to reason about objects outside the field of view. Moreover, the model is capable of acquiring new symbols in an online fashion. The effectiveness of the model is evaluated via simulations and real experiments in terms of grounding and learning new phrases and objects.

%Research in automatic natural language grounding, the problem of robots associating phrases with realworld objects and actions, offers a tantalizing reality in which untrained humans can operate sophisticated robots. 

%Current techniques for training robots to understand natural language, however, assume that there is a fixed set of phrases or objects that the robot will encounter during deployment. 

%Instead, the real world is full of confusing jargon and unique objects that are nearly impossible to anticipate and therefore train for. 
%This paper presents a model called the Distributed Correspondence Graph - Unknown Phrase, Unknown Percept - Away (DCGUPUP- Away) that augments a previously successful model by explicitly representing unknown phrases and objects as unknown, as well as reasoning about objects that are not currently perceived. 
%Furthermore, experimental results in simulation,
%as well as a trial run on a physical turtlebot, validate the
%effectiveness of DCG-UPUP-Away in grounding and learning
%new phrases.